

\documentclass[11pt, oneside]{article}   	% use "amsart" instead of "article" for AMSLaTeX format
\usepackage{geometry}                		% See geometry.pdf to learn the layout options. There are lots.
\geometry{letterpaper}                   		% ... or a4paper or a5paper or ... 
%\geometry{landscape}                		% Activate for rotated page geometry
\usepackage[parfill]{parskip}    		% Activate to begin paragraphs with an empty line rather than an indent
\usepackage{graphicx}				% Use pdf, png, jpg, or eps§ with pdflatex; use eps in DVI mode
								% TeX will automatically convert eps --> pdf in pdflatex		
\usepackage{amssymb}
\usepackage{amsthm}
\usepackage{amsmath} 
\usepackage{bm}
\usepackage{xcolor}
\usepackage{bbm}
\usepackage{bbold}
\usepackage[T1]{fontenc}
\usepackage{subfigure}
\usepackage[english]{babel}
\newtheorem{theorem}{Theorem}[subsection]
\newtheorem{corollary}{Corollary}[theorem]
\newtheorem{lemma}[theorem]{Lemma}
\newtheorem{mydef}{Definition}

\usepackage[utf8]{inputenc}


%SetFonts

%SetFonts


\title{Proof for Lemma 2.1}
\author{}
\date{}							% Activate to display a given date or no date

\begin{document}
\maketitle
\section{Notations}
\begin{itemize}
	\item {$N$: Number of users}
	
	\item {$n$: Number of items,    $
		n = \left \{
		\begin{aligned}
		&2m-1, && \text{if n is odd} \\
		&2m, && \text{otherwise}
		\end{aligned} \right.
		$}
	
	\item{$\mathcal{P}_n$: the space of permutation of $n$ items}
	\item {$\bm{R}^1,...,\bm{R}^N$: full rankings given by the users, $\bm{R}_j \sim \text{Mallows}(\alpha^0, \bm{\rho}^0)$}
	
	\item{ \{$o_1, ..., o_n$\}: an ordering of $n$ items that corresponds to a ranking \{$i_1, ..., i_n$\} s.t. $i_{o_j} = j$ }.
\end{itemize}

\subsection{}
\begin{lemma} 
	For any given $n$ and $\alpha^0>0$, $\mathbb{E}[{R}_{o_j}]<\mathbb{E}[{R}_{o_{j+1}}]$, where $\{\rho_{o_1}, ..., \rho_{o_n}\} = \{1, 2, ..., n\}$
	
\end{lemma}

\begin{proof}
According to the definition of expectations, \\
$\mathbb{E}[R_{o_j}] = \sum\limits_{\bm{r}\in \mathcal{P}_n} r_{o_j}\text{exp}\{-\frac{\alpha^0}{n}d(\bm{r}, \bm{\rho^0})\}$, and  $\mathbb{E}[R_{o_{j+1}}] = \sum\limits_{\bm{r}\in \mathcal{P}_n} r_{o_{j+1}}\text{exp}\{-\frac{\alpha^0}{n}d(\bm{r}, \bm{\rho^0})\}$ \\
Given a permutation $r$ s.t. $r_{o_j} = a$ and  $r_{o_{j+1}} = b$, $a<b$, we can find another permutation $r'$ s.t. $r'_{o_{j+1}} = a$ and  $r_{o_{j}} = b$, $r_i = r'_i$, $\forall i\neq o_j, o_{j+1}$.

If we divide $\mathcal{P}_n $ into
\begin{enumerate}
	\item{$\mathcal{P}_A$ = $\{r: r_{o_j} = a, r_{o_{j+1}=b}\}$, $\mathcal{P}_{A'}$ = $\{r': r'_{o_j} = b, r'_{o_{j+1}=a}\}$, $a\geq j+1$}
	
	\item{$\mathcal{P}_B$ = $\{r: r_{o_j} = a, r_{o_{j+1}=b}\}$, $\mathcal{P}_{B'}$ = $\{r': r'_{o_j} = b, r'_{o_{j+1}=a}\}$, $b\leq j$}
	
	\item{$\mathcal{P}_C$ = $\{r: r_{o_j} = a, r_{o_{j+1}=b}\}$, $\mathcal{P}_{C'}$ = $\{r': r'_{o_j} = b, r'_{o_{j+1}=a}\}$, $a\leq j$, $b\geq j+1$},
\end{enumerate}

we can rewrite the expectations to be: \\
$\mathbb{E}[R_{o_{j}}] = \sum\limits_{\bm{r}\in \mathcal{P}_A} r_{o_j}\text{exp}\{-\frac{\alpha^0}{n}d(\bm{r}, \bm{\rho^0})\}$ +$\sum\limits_{\bm{r}\in \mathcal{P}_{A'}} r_{o_j}\text{exp}\{-\frac{\alpha^0}{n}d(\bm{r}, \bm{\rho^0})\}$ +\\$\sum\limits_{\bm{r}\in \mathcal{P}_B} r_{o_j}\text{exp}\{-\frac{\alpha^0}{n}d(\bm{r}, \bm{\rho^0})\}$ +$\sum\limits_{\bm{r}\in \mathcal{P}_{B'}} r_{o_j}\text{exp}\{-\frac{\alpha^0}{n}d(\bm{r}, \bm{\rho^0})\}$+ \\$\sum\limits_{\bm{r}\in \mathcal{P}_C} r_{o_j}\text{exp}\{-\frac{\alpha^0}{n}d(\bm{r}, \bm{\rho^0})\}$+$\sum\limits_{\bm{r}\in \mathcal{P}_{C'}} r_{o_j}\text{exp}\{-\frac{\alpha^0}{n}d(\bm{r}, \bm{\rho^0})\}$, and similarly,

$\mathbb{E}[R_{o_{j+1}}] = \sum\limits_{\bm{r}\in \mathcal{P}_A} r_{o_{j+1}}\text{exp}\{-\frac{\alpha^0}{n}d(\bm{r}, \bm{\rho^0})\}$ +$\sum\limits_{\bm{r}\in \mathcal{P}_{A'}} r_{o_{j+1}}\text{exp}\{-\frac{\alpha^0}{n}d(\bm{r}, \bm{\rho^0})\}$ +\\$\sum\limits_{\bm{r}\in \mathcal{P}_B} r_{o_{j+1}}\text{exp}\{-\frac{\alpha^0}{n}d(\bm{r}, \bm{\rho^0})\}$ +$\sum\limits_{\bm{r}\in \mathcal{P}_{B'}} r_{o_{j+1}}\text{exp}\{-\frac{\alpha^0}{n}d(\bm{r}, \bm{\rho^0})\}$+ \\$\sum\limits_{\bm{r}\in \mathcal{P}_C} r_{o_{j+1}}\text{exp}\{-\frac{\alpha^0}{n}d(\bm{r}, \bm{\rho^0})\}$+$\sum\limits_{\bm{r}\in \mathcal{P}_{C'}} r_{o_{j+1}}\text{exp}\{-\frac{\alpha^0}{n}d(\bm{r}, \bm{\rho^0})\}$.

Let us first consider $\mathcal{P}_A$ and $\mathcal{P}_{A'}$'s contributions to $\mathbb{E}[R_{o_{j}}]$ and $\mathbb{E}[R_{o_{j+1}}]$.
For any $r \in \mathcal{P}_A$ and its corresponding $r' \in \mathcal{P}_{A'}$ s.t. $r_{o_j} = r'_{o_{j+1}} = a$, $r_{o_{j+1}} = r'_{o_{j}} = b$, and $r_i = r'_i \forall i\neq o_j, o_{j+1}$, it can be inferred that $P(\bm{r}) = P(\bm{r'})$, since \\
$d(\bm{r}, \bm{\rho^0}) = \sum\limits_{i=1}^{j-1}|r_{o_i} - \rho^0_{o_i}| + |r_{o_j} - \rho_{o_j}| + |r_{o_{j+1}} - \rho^0_{o_{j+1}}| + \sum\limits_{i = j+2}^{n}|r_{o_i} - \rho^0_{o_{i}}|$ \\
= $\sum\limits_{i \neq j, j+1} |r_{o_i} - i| + |a-j| + |b - (j+1)|$\\
=$\sum\limits_{i \neq j, j+1} |r_{o_i} - i| + a-j + b - (j+1)$, similarly, \\

$d(\bm{r'}, \bm{\rho^0}) = \sum\limits_{i=1}^{j-1}|r'_{o_i} - \rho^0_{o_i}| + |r'_{o_j} - \rho_{o_j}| + |r'_{o_{j+1}} - \rho^0_{o_{j+1}}| + \sum\limits_{i = j+2}^{n}|r'_{o_i} - \rho^0_{o_{i}}|$ \\
= $\sum\limits_{i \neq j, j+1} |r'_{o_i} - i| + |b-j| + |a - (j+1)|$\\
=$\sum\limits_{i \neq j, j+1} |r_{o_i} - i| + b-j + a - (j+1)$ \\
= $d(\bm{r}, \bm{\rho^0}) $

For each \{$\bm{r}, \bm{r'}$\} pair, their contributions to $\mathbb{E}[R_{o_j}]$ and $\mathbb{E}[R_{o_{j+1}}]$ are: \\
$\mathbb{E}[R_{o_j}]|_{\bm{r}\in \mathcal{P}_A,\bm{r'}\in \mathcal{P}_{A'}} = P(\bm{r})\cdot r_{o_j} + P(\bm{r'})\cdot r'_{o_j} = P(\bm{r})\cdot (a+b) $ \\

$\mathbb{E}[R_{o_{j+1}}]|_{\bm{r}\in \mathcal{P}_A,\bm{r'}\in \mathcal{P}_{A'}} = P(\bm{r})\cdot r_{o_{j+1}} + P(\bm{r'})\cdot r'_{o_{j+1}} = P(\bm{r})\cdot (a+b)$ 

Therefore, for all $\bm{r}\in \mathcal{P}_A$ and their corresponding $\bm{r'}\in \mathcal{P}_{A'}$, we have \\
$\sum\limits_{\bm{r}\in \mathcal{P}_A, \bm{r}\in \mathcal{P}_{A'}}\mathbb{E}[R_{o_j}] |_{\bm{r}, \bm{r'}} = \sum\limits_{\bm{r}\in \mathcal{P}_A, \bm{r}\in \mathcal{P}_{A'}}\mathbb{E}[R_{o_{j+1}}] |_{\bm{r}, \bm{r'}}$

Similarly, let us now consider $\mathcal{P}_B$ and $\mathcal{P}_{B'}$'s contributions to $\mathbb{E}[R_{o_{j}}]$ and $\mathbb{E}[R_{o_{j+1}}]$.

For any $r \in \mathcal{P}_B$ and its corresponding $r' \in \mathcal{P}_{B'}$ s.t. $r_{o_j} = r'_{o_{j+1}} = a$, $r_{o_{j+1}} = r'_{o_{j}} = b$, and $r_i = r'_i \forall i\neq o_j, o_{j+1}$, it can be inferred that $P(\bm{r}) = P(\bm{r'})$, since \\
$d(\bm{r}, \bm{\rho^0}) = \sum\limits_{i=1}^{j-1}|r_{o_i} - \rho^0_{o_i}| + |r_{o_j} - \rho_{o_j}| + |r_{o_{j+1}} - \rho^0_{o_{j+1}}| + \sum\limits_{i = j+2}^{n}|r_{o_i} - \rho^0_{o_{i}}|$ \\
= $\sum\limits_{i \neq j, j+1} |r_{o_i} - i| + |a-j| + |b - (j+1)|$\\
=$\sum\limits_{i \neq j, j+1} |r_{o_i} - i| + j-a + (j+1)-b$, and \\

$d(\bm{r'}, \bm{\rho^0}) = \sum\limits_{i=1}^{j-1}|r'_{o_i} - \rho^0_{o_i}| + |r'_{o_j} - \rho_{o_j}| + |r'_{o_{j+1}} - \rho^0_{o_{j+1}}| + \sum\limits_{i = j+2}^{n}|r'_{o_i} - \rho^0_{o_{i}}|$ \\
= $\sum\limits_{i \neq j, j+1} |r'_{o_i} - i| + |b-j| + |a - (j+1)|$\\
=$\sum\limits_{i \neq j, j+1} |r_{o_i} - i| + j-b + (j+1)- a$ \\
= $d(\bm{r}, \bm{\rho^0}) $

For each \{$\bm{r}, \bm{r'}$\} pair, their contributions to $\mathbb{E}[R_{o_j}]$ and $\mathbb{E}[R_{o_{j+1}}]$ are: \\
$\mathbb{E}[R_{o_j}]|_{\bm{r}\in \mathcal{P}_B,\bm{r'}\in \mathcal{P}_{B'}} = P(\bm{r})\cdot r_{o_j} + P(\bm{r'})\cdot r'_{o_B} = P(\bm{r})\cdot (a+b) $ \\

$\mathbb{E}[R_{o_{j+1}}]|_{\bm{r}\in \mathcal{P}_B,\bm{r'}\in \mathcal{P}_{B'}} = P(\bm{r})\cdot r_{o_{j+1}} + P(\bm{r'})\cdot r'_{o_{j+1}} = P(\bm{r})\cdot (a+b)$ 

Therefore, for all $\bm{r}\in \mathcal{P}_A$ and their corresponding $\bm{r'}\in \mathcal{P}_{A'}$, we have \\
$\sum\limits_{\bm{r}\in \mathcal{P}_B, \bm{r}\in \mathcal{P}_{B'}}\mathbb{E}[R_{o_j}] |_{\bm{r}, \bm{r'}} = \sum\limits_{\bm{r}\in \mathcal{P}_B, \bm{r}\in \mathcal{P}_{B'}}\mathbb{E}[R_{o_{j+1}}] |_{\bm{r}, \bm{r'}}$

Up to this point, we have proven that \\
$\sum\limits_{\bm{r}\in \mathcal{P}_A, \bm{r}\in \mathcal{P}_{A'}}\mathbb{E}[R_{o_j}] |_{\bm{r}, \bm{r'}} + \sum\limits_{\bm{r}\in \mathcal{P}_B, \bm{r}\in \mathcal{P}_{B'}}\mathbb{E}[R_{o_j}] |_{\bm{r}, \bm{r'}}= \sum\limits_{\bm{r}\in \mathcal{P}_A, \bm{r}\in \mathcal{P}_{A'}}\mathbb{E}[R_{o_{j+1}}] |_{\bm{r}, \bm{r'}}+\sum\limits_{\bm{r}\in \mathcal{P}_B, \bm{r}\in \mathcal{P}_{B'}}\mathbb{E}[R_{o_{j+1}}] |_{\bm{r}, \bm{r'}}$.

For $\alpha > 0$, to prove that $\mathbb{E}[R_{o_j}] < \mathbb{E}[R_{o_{j+1}}]$ is equivalent to proving \\
$\sum\limits_{\bm{r}\in \mathcal{P}_C, \bm{r}\in \mathcal{P}_{C'}}\mathbb{E}[R_{o_j}] |_{\bm{r}, \bm{r'}} < \sum\limits_{\bm{r}\in \mathcal{P}_C, \bm{r}\in \mathcal{P}_{C'}}\mathbb{E}[R_{o_{j+1}}] |_{\bm{r}, \bm{r'}}$.


Now let us consider $\mathcal{P}_C$ and $\mathcal{P}_{C'}$'s contributions to $\mathbb{E}[R_{o_{j}}]$ and $\mathbb{E}[R_{o_{j+1}}]$.


For any $r \in \mathcal{P}_C$ and its corresponding $r' \in \mathcal{P}_{C'}$, we have

$d(\bm{r}, \bm{\rho^0}) = \sum\limits_{i=1}^{j-1}|r_{o_i} - \rho^0_{o_i}| + |r_{o_j} - \rho_{o_j}| + |r_{o_{j+1}} - \rho^0_{o_{j+1}}| + \sum\limits_{i = j+2}^{n}|r_{o_i} - \rho^0_{o_{i}}|$ \\
= $\sum\limits_{i \neq j, j+1} |r_{o_i} - i| + |a-j| + |b - (j+1)|$\\
=$\sum\limits_{i \neq j, j+1} |r_{o_i} - i| + j-a + b-(j+1)$\\
=$\sum\limits_{i \neq j, j+1} |r_{o_i} - i| + b-a -1 $, and \\

$d(\bm{r'}, \bm{\rho^0}) = \sum\limits_{i=1}^{j-1}|r'_{o_i} - \rho^0_{o_i}| + |r'_{o_j} - \rho_{o_j}| + |r'_{o_{j+1}} - \rho^0_{o_{j+1}}| + \sum\limits_{i = j+2}^{n}|r'_{o_i} - \rho^0_{o_{i}}|$ \\
= $\sum\limits_{i \neq j, j+1} |r'_{o_i} - i| + |b-j| + |a - (j+1)|$\\
=$\sum\limits_{i \neq j, j+1} |r_{o_i} - i| + b-j + (j+1)- a$ \\
= $\sum\limits_{i \neq j, j+1} |r_{o_i} - i| + b-a+1 $\\
=$d(\bm{r}, \bm{\rho^0}) +2$

Therefore, for any $\alpha^0 > 0$ and any given $\bm{r} \in \mathcal{P}_C$ and its corresponding $\bm{r'} \in \mathcal{P}_{C'}$, we have \\
 $P(\bm{r}) > P(\bm{r'}) $. 
 
 For each \{$\bm{r}, \bm{r'}$\} pair, their contributions to $\mathbb{E}[R_{o_j}]$ and $\mathbb{E}[R_{o_{j+1}}]$ are: \\
$\mathbb{E}[R_{o_j}]|_{\bm{r}\in \mathcal{P}_C,\bm{r'}\in \mathcal{P}_{C'}} = P(\bm{r})\cdot r_{o_j} + P(\bm{r'})\cdot r'_{o_B} = a\cdot P(\bm{r})\cdot + b\cdot P(\bm{r'}) $ \\
$\mathbb{E}[R_{o_{j+1}}]|_{\bm{r}\in \mathcal{P}_C,\bm{r'}\in \mathcal{P}_{C'}} = P(\bm{r})\cdot r_{o_{j+1}} + P(\bm{r'})\cdot r'_{o_{j+1}} = b\cdot P(\bm{r})\cdot + a\cdot P(\bm{r'})$ 

$\mathbb{E}[R_{o_j}]|_{\bm{r}\in \mathcal{P}_C,\bm{r'}\in \mathcal{P}_{C'}} - \mathbb{E}[R_{o_{j+1}}]|_{\bm{r}\in \mathcal{P}_B,\bm{r'}\in \mathcal{P}_{B'}}$ \\ =$(a-b)P(\bm{r}) + (b-a)P(\bm{r'})$ \\
= $(a-b)\cdot (P(\bm{r}) - P(\bm{r'}))$

Recall that $a<b$ and $P(\bm{r}) > P(\bm{r'})$ for any $r \in \mathcal{P}_C$ and its corresponding $r' \in \mathcal{P}_{C'}$, it can be obtained that: \\
$\mathbb{E}[R_{o_j}]|_{\bm{r}\in \mathcal{P}_C,\bm{r'}\in \mathcal{P}_{C'}} - \mathbb{E}[R_{o_{j+1}}]|_{\bm{r}\in \mathcal{P}_B,\bm{r'}\in \mathcal{P}_{B'}} < 0$, i.e., $\mathbb{E}[R_{o_j}]|_{\bm{r}\in \mathcal{P}_C,\bm{r'}\in \mathcal{P}_{C'}} < \mathbb{E}[R_{o_{j+1}}]|_{\bm{r}\in \mathcal{P}_B,\bm{r'}\in \mathcal{P}_{B'}}$

It is straight forward to obtain that:\\
 $\sum\limits_{\bm{r}\in \mathcal{P}_C, \bm{r}\in \mathcal{P}_{C'}}\mathbb{E}[R_{o_j}] |_{\bm{r}, \bm{r'}} < \sum\limits_{\bm{r}\in \mathcal{P}_C, \bm{r}\in \mathcal{P}_{C'}}\mathbb{E}[R_{o_{j+1}}] |_{\bm{r}, \bm{r'}}$,
 
and therefore, 
for any $\alpha > 0$, we have $\mathbb{E}[R_{o_j}] < \mathbb{E}[R_{o_{j+1}}]$

\end{proof}

\end{document}